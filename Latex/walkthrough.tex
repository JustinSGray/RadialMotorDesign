\documentclass[10pt]{article}
\usepackage[margin=1in]{geometry}
\usepackage[english]{babel}
\newcommand{\ignore}[1]{}
\usepackage[utf8]{inputenc}

\usepackage{graphicx}
\usepackage{amsmath}
\usepackage[version=4]{mhchem}
\usepackage{siunitx}
\usepackage{longtable,tabularx}
\usepackage{array}

\title{PMSM Walkthrough}
\author{ }
\begin{document}

\maketitle

\section{Electromagnetic Group}

\subsection{CartersComp}

\begin{equation}
    angle_{m} = \frac{4}{\pi}*\Bigg(\bigg((\frac{0.5 * slot\_o}{g})*\arctan(\frac{0.5*slot\_o}{g})\bigg) - \bigg(ln\Big(\sqrt{(1+((0.5*slot\_o/g)))}\Big)\bigg)\Bigg)
\end{equation}


\begin{equation}
    pitch = \frac{\pi*R_{in,s}}{n_{slots}}
\end{equation}

\begin{equation}
    C =\frac{pitch}{pitch - angle_{m} * g}
\end{equation}

Where C is Carters Coefficient, and is passed into the Gap Equivalent Component.

\subsection{GapEquivalentComp}

\begin{equation}
    g_{eq} = g * C * k_{sat}
\end{equation}
Where $k_{sat}$ is the saturation factor of the magnetic circuit due to the main (linkage) magnetic flux. The equivalent gap is used to calculate the air gap flux density.

\subsection{GapFieldsComp}
I need to include the temperature dependant flux density and field intensity calculations. Already typed up, just include them now.

Flux density calculation:
\begin{equation}
    B_{g} = \frac{Br_{20} * t_{mag}}{(1+mu_{r}*g_{eq})}
\end{equation}
Where $Br_{20}$ is the remnance flux density at 20$^\circ$C, $t_{mag}$ is the operating temperature of the magnet, $mu_{r}$ is the relative magnetic permeability of ferromagnetic materials, equal to 1.04 H/m. Flux density is used in the field intensity calculation and for the size component to calculate the geometry of the motor.

Field intensity Calcuation:
\begin{equation}
    H_{g} = Hc_{20} * \frac{B_g}{Br_{20}}
\end{equation}
Where $Hc_{20}$ is the intrinsic coercivity at 20$^\circ$C. Only the flux density, $B_{g}$, is passed on to the torque component, seen next.

\subsection{Torque}

In order to calculate torque, the volume of the rotor and the stator surface current need to be calculated. The volume of the rotor can be seen here in equation \ref{eqn:vrot}
\begin{equation}
    V_{rotor} = \pi * R_{out,r}^2 *L_{stack}
    \label{eqn:vrot}
\end{equation}
Where $R_{out,r}$ is the radius of the outer portion of the rotor, and $L_{stack}$ is the stack length, or axial length, of the motor.

The surface current of the stator can be seen below.
\begin{equation}
    I_{stator} = 6 * 0.645* \frac{96}{2\pi R_{in,s}} * I \sqrt{2}
    \label{eqn:istator}
\end{equation}
Where I is the RMS value of the current for the motor.

\begin{equation}
    T_q = V_{rotor} * B_g * I_{stator} * \cos{(\epsilon)}
    \label{eqn:tq}
\end{equation}
The torque equation applies directly to the pmsm type of motor. If other types of motors are used, the torque equation should reflect that change appropriately. Torque is then passed into motor efficiency. 

\subsection{EfficiencyComp}
The efficiency component should take in a fully developed motor and produce a table and plot for mapping the efficiency. The inputs and outputs should be compatible with ccblade and zappy.


\textbf{Inputs:}
\begin{enumerate}
    \item Current
    \item Torque
    \item Voltage
    \item RPM
\end{enumerate}

\textbf{Outputs:}
\begin{enumerate}
    \item Efficiency
    \item Shaft Power
    \item Input Power
\end{enumerate}

\section{Thermal Group}

This sectio consists of two components, copper losses and iron losses. The copper losses are not complete yet. The iron losses are computed using a Steinmetz loss model, with the coefficients found by curve fitting existing measured data for Hiperco-50. Iron losses should include losses from non-uniform waveforms and the additional eddy currents.


\subsection{Copper Loss (WindingLossComp)}
This component outputs the following:
\begin{enumerate}
    \item Wire Length
    \item Temp Dependent Resistivity
    \item DC resistance
    \item Skin Depth
    \item AC Resistance
    \item Copper Losses
\end{enumerate}
Note that all current values are RMS values, not peak current.

\begin{equation}
    L_{wire} = \frac{n_{slots}*n_{turns}}{3} * 2L_{stack} + L_{endturns}
\end{equation}
Where $n_{slots}$ is the number of slots, $n_{turns}$ are the number of wire turns, and $L_{endturns}$ are the length of the end windings, accounting for both sides of the motor. The number three represents the number of phases.

The temperature dependent resistivity of copper is calculated here.
\begin{equation}
    Rs_{temp} = (Rs_{wire} * (1+T_{coef}*T\_windings))
\end{equation}
Where $Rs_{wire}$ is the resisitivity of Cu at 20$^\circ$C, $T_{coef}$ is the temperature coefficient for copper, T\_windings is the assumed operating temperature of windings. The $Rs_{wire}$ is passed to the DC resistance calculation, and the skin depth calculation.

\begin{equation}
    R_{dc} = \frac{Rs_{temp}*L_{wire}}{\pi*r_{litz}^2}
\end{equation}
Where $r_{litz}$ is the radius of the litz wire strand. Is actually $r_{wire}$ in code, should change for readability.
\begin{equation}
    skin\_depth = \sqrt{ \frac{Rs_{temp}}{\pi*f_e*\mu_r*\mu_0}}
\end{equation}
Where $f_{e}$ is the electrical frequency. Is currently an input but should be an output dependant on the RPM and motor config.
\begin{equation}
    R_{ac} = \frac{R_{dc}}{\frac{2*skin\_depth}{r_{strand}} - (\frac{skin\_depth}{r_{litz}})^{2}}
\end{equation}
Where $r_{strand}$ is the radius of one strand of litz wire. This resistance value is passed into the $I^{2}R$ losses for the windings, seen next.
\begin{equation}
    P_{cu} = \Big(\frac{I}{3}\Big)^2  R_{ac}
\end{equation}
Where I is the RMS value of current.


\subsection{Iron Losses (SteinmetzLossComp)}
This uses the simple version of the Steinmetz to calculate iron losses. the coefficients used were generated using a surface fit for Hiperco-50, which allows for the coefficients to be constant for all frequency ranges.

\begin{equation}
    P_{iron} = K * f_e^{\alpha} * B_{pk}^{\beta}
\end{equation}
Where K, $\alpha$, and $\beta$ are the constant coefficients for the Steinmetz equation.

\section{Geometry Group}
This section produces the motor geometry and motor mass, excluding the copper currently. The sizing component outputs the following:
\begin{enumerate}
    \item width of stator yoke
    \item width of tooth
    \item width of stator yoke
    \item slot depth
    \item rotor inner radius
    \item stator inner radius
    \item area of one slot
    \item Current density of the wires
\end{enumerate}

All sizes are based on a pre-selected current density. The current density can be used in place of thermal analysys as a rule of thumb to prescribe a maximun amount of heat generation in the windings. Once the size is calculated, mass can be calculated. The motor mass component outputs the following:
\begin{enumerate}
    \item stator mass
    \item rotor mass
    \item magnet mass
\end{enumerate}
Copper mass needs to be added here as well.

\subsection{Motor Sizing (MotorSizeComp)}
This section shows the equations used in calculating specific geometries of the motor. Flux density values for all equations are user defined values, but should be incorporated with a FEA method to increase accuracy. The values used are the max flux densities the materials can operate at.

The stator yoke width:
\begin{equation}
    w_{ry} = \frac{\pi*R_{out,r}*B_{g}}{k*N_{poles}*B_{ry}}
\end{equation}
Where k is the stacking factor, $N_{poles}$ is the number of poles, and $B_{ry}$ is the rotor yoke flux density.

Width of the stator tooth:
\begin{equation}
    w_t = \frac{2\pi*R_{out, r}*B_g}  { N_{slots*k*B_t}}
\end{equation}
where $B_t$ is the tooth flox density

Width of the stator yoke:
\begin{equation}
    w_sy = \frac{\pi*R_{out, r*B_g}}  { N_{poles*k*B_sy}}
\end{equation}
Where $B_{sy}$ is the flux density of the stator yoke.

Slot depth:
\begin{equation}
    slot_d = R_{motor}- R_{out, r} - g - w_sy
\end{equation}

Rotor inner radius:
\begin{equation}
    R_{in, r} = R_{out, r} - t_{mag} - w_{ry}
\end{equation}
Where $t_{mag}$ is the radial thickness of the magnets.

Stator Inner Radius
\begin{equation}
    R_{in, s} = R_{out, r} + g
\end{equation}

Slot area:
\begin{equation}
    A_{slot} = \pi * (R_{motor} - w_{sy})^{2} - \pi(R_{motor}-w_sy - slot_d)^{2}
\end{equation}

Current density is defined by the amount of current
\begin{equation}
    J = \frac{2*N_{turns}*N_{slots}*I*(\sqrt{2})}{k_{wb} * \frac{1}{(A_{slot}-N_{slots}*w_t*s_d)}}*1E6
\end{equation}
Where $k_{wb}$ is the bare wire slot fill factor and the 1E6 term is to convert to units of $A/mm^2$. The current density is set by the user, and this equation is used in the balance which solves for the rotor outer radius, then all other geometry defined above is calculated.


\subsection{Motor Mass (MotorMassComp)}
After the geometry is calculated, the component mass can then be calculated for the stator, rotor, and magnets. Copper to be added soon.

Stator mass:
\begin{equation}
    M_{stator} = \rho_{stator}*L_{stack} \Big((\pi R_{motor}^2) - \pi(R_{in,s}+slot_d)^2+(N_{slots}*w_t*slot_d)\Big)
\end{equation}
Where $\rho_{stator}$ is the density of the iron used in the stator.

Rotor Mass:
\begin{equation}
    M_{rotor} = \big(\pi (R_{out,r} - t_{mag})^2 - \pi*R_{out,r}^2\big) * \rho_{rotor}  L_{stack}
\end{equation}
Where $\rho_{rotor}$ is the density of the iron used in the rotor.

Magnet Mass:
\begin{equation}
    M_{mag} = \Big((\pi*R_{out,r}^2) - \pi*(R_{out,r}-t_{mag})^2\Big) * \rho_{mag} L_{stack}
\end{equation}
Where $\rho_{mag}$ is the density of the magnets.



\end{document}

